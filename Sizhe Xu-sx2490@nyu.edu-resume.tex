%------------------------
% Resume Template
% Author : Anubhav Singh
% Github : https://github.com/xprilion
% License : MIT
%------------------------
%! TeX root = "Sizhe Xu-sx2490@nyu.edu-resume.tex"
\documentclass[a4paper,20pt]{article}

\usepackage{latexsym}
\usepackage[empty]{fullpage}
\usepackage{titlesec}
\usepackage{marvosym}
\usepackage[usenames,dvipsnames]{color}
\usepackage{verbatim}
\usepackage{enumitem}
\usepackage{fontawesome5}
\usepackage[pdftex]{hyperref}
\usepackage{fancyhdr}
\usepackage{forest}
\usepackage{academicons}

% Configure hyperref to remove colored boxes around links
\hypersetup{
    hidelinks,
    colorlinks=true,
    linkcolor=black,
    filecolor=black,
    urlcolor=black,
    citecolor=black
}

\pagestyle{fancy}
\fancyhf{} % clear all header and footer fields
\fancyfoot{}
\renewcommand{\headrulewidth}{0pt}
\renewcommand{\footrulewidth}{0pt}

% Adjust margins
\addtolength{\oddsidemargin}{-0.530in}
\addtolength{\evensidemargin}{-0.375in}
\addtolength{\textwidth}{1in}
\addtolength{\topmargin}{-.45in}
\addtolength{\textheight}{1in}

\urlstyle{rm}

\raggedbottom
\raggedright
\setlength{\tabcolsep}{0in}

% Sections formatting
\titleformat{\section}{
  \vspace{-10pt}\scshape\raggedright\large
}{}{0em}{}[\color{black}\titlerule \vspace{-6pt}]

%-------------------------
% Custom commands
\newcommand{\github}[1]{
  \href{#1}{\faGithub}
}

\newcommand{\paper}[1]{
  % \href{#1}{\faScroll}
  \href{#1}{\faLink}
}

\newcommand{\resumeItem}[2]{
  \item\small{
    \textbf{#1}{: #2 \vspace{-2pt}}
  }
}

\newcommand{\resumePubItem}[1]{
  \item\small{#1 \vspace{1pt}} 
}

\newcommand{\resumeProjectItem}[2]{
  \item\small{
    \textbf{#1}{\\ \vspace{1pt} #2 \vspace{-3pt}}
  }
}

\newcommand{\resumeProjectTitle}[3][]{
  \vspace{-1pt}\item
  \begin{tabular*}{0.97\textwidth}{l@{\extracolsep{\fill}}r}
    \textbf{#2} #1 & #3
  \end{tabular*}\vspace{1pt}
}

\newcommand{\resumeItemContent}[2]{
  {
    #1\\ \vspace{1pt} \textbf{Tech: }#2\vspace{-3pt}
  }
}

\newcommand{\resumeSubheading}[4]{
  \vspace{-1pt}\item
    \begin{tabular*}{0.97\textwidth}{l@{\extracolsep{\fill}}r}
      \textbf{#1} & #2 \\
      {#3} & {#4} \\
    \end{tabular*}\vspace{-5pt}
}

\newcommand{\resumeSubItem}[2]{\resumeItem{#1}{#2}\vspace{-3pt}}

\renewcommand{\labelitemii}{$\circ$}

% 添加一个处理长标题的特殊命令
\newcommand{\resumeLongTitleSubheading}[4]{
  \vspace{-1pt}\item
    \begin{tabular*}{0.97\textwidth}{l@{\extracolsep{\fill}}r}
      \multicolumn{2}{l}{\textbf{#1}} \\
      {#3} & {#4} \\
    \end{tabular*}\vspace{-5pt}
}

\newcommand{\resumeSubHeadingListStart}{\begin{itemize}[leftmargin=0pt, label={}]}
\newcommand{\resumeSubHeadingListEnd}{\end{itemize}}
\newcommand{\resumeItemListStart}{\begin{itemize}}
\newcommand{\resumeItemListEnd}{\end{itemize}\vspace{-5pt}}

%-----------------------------
%%%%%%  CV STARTS HERE  %%%%%%

\begin{document}

%----------HEADING-----------------
\begin{tabular*}{\textwidth}{l@{\extracolsep{\fill}}r@{\hspace{-17em}}l}
  \textbf{{\LARGE Sizhe (Alex) Xu}} & Github: & MazelTovy \github{https://github.com/MazelTovy}\\
  Website: ~\href{https://sizhexu.com}{sizhexu.com} & Email: & \href{mailto:sx2490@nyu.edu}{sx2490@nyu.edu}~~~~~ \\
  %\href{https://github.com/MazelTovy}{Github: ~~github.com/MazelTovy\ \faGithubSquare} \\
  % \href{https://github.com/MazelTovy}{Github: ~~MazelTovy \github{https://github.com/MazelTovy}} & Mobile: & (347)-712-0812 \\
  Location: 370 Jay St, Brooklyn, New York, 11201 & Mobile: & (347)-712-0812~~~~~ \\
\end{tabular*}

%-----------EDUCATION-----------------
\vspace{3pt}
\section{Education}
  \resumeSubHeadingListStart
    \resumeSubheading
      {New York University}{Sept 2024 - June 2026}
      {Master of Science - Urban Data Science}{}
    \resumeItemListStart
        \resumeItem{Courses}{Urban Computing \& AI, Data Science, Deep Learning, Computer Vision, Large Language and Vision Models, Transportation and Logistics, Innovative City Governance, Probability and Stochastic Processes}
    \resumeItemListEnd
    \resumeSubheading
      {Dalian Jiaotong University}{Sept 2020 - June 2024}
      {Bachelor of Engineering - Electronic Engineering}{}
    \resumeItemListStart
        \resumeItem{Courses}{Analog Electronics, Object-Oriented Programming, Algorithm Design, Machine Learning}
    \resumeItemListEnd
  \resumeSubHeadingListEnd

%-----------PUBLICATIONS-----------------
\vspace{-3pt}
\section{Publications}
\begin{enumerate}[leftmargin=*, label={[\arabic*].}, nosep]
  \resumePubItem{\textbf{Sizhe Xu}{$^\ast$}, Renzhao Liang{$^\ast$}, Chenggang Xie, Jingru Chen, Feiyang Ren, Shu Yang, Takahiro Yabe, "Abstain Mask Retain Core: Time Series Prediction by Adaptive Masking Loss", \textit{Advances in Neural Information Processing Systems} (2025) \textbf{Spotlight (top 5\%)}. {$^\ast$} Denotes equal contribution.}
  \resumePubItem{\textbf{Sizhe Xu}, Renzhao Liang, Jun Han, Qitong Sun, "A Hybrid Framework for Evaluating and Enhancing Syntactic and Semantic Diversity in Low-Resource Text Generation", \textit{Under Review} (2025).}
  \resumePubItem{\textbf{Sizhe Xu}, Boyang Li, Donghak Lee, Takahiro Yabe, "Thinking on the Move (ToM): A Framework for LLM-Agent-based Reinforcement Learning in Urban Mobility Simulation", \textit{In Progress} (2025).}
  \resumePubItem{Boyang Li, \textbf{Sizhe Xu}, Yulin Wu, Takahiro Yabe, "A Generalized RoPE for \textit{n}-Dimensional Position Embedding", \textit{In Progress} (2025).}
\end{enumerate}

%-----------RESEARCH-----------------
\vspace{1pt}
\section{Research}
  \resumeSubHeadingListStart
    \resumeLongTitleSubheading
      {Abstain Mask Retain Core: Time Series Prediction by Adaptive Masking Loss \paper{https://arxiv.org/abs/2510.19980}}{}
      {Co-first Author, Neural Information Processing Systems (NeurIPS) 2025 Spotlight}{Mar 2025 - Aug 2025}
    \resumeItemListStart
      \resumeItem{Theoretical Innovation}
      {Challenged the prevailing long-sequence information gain hypothesis by identifying a counterintuitive phenomenon where strategic input truncation enhances forecasting accuracy. Leveraged the Information Bottleneck theory to formalize the trade-off between compression and prediction, proposing that explicit redundancy reduction in the latent space is critical for effective temporal signal extraction.}
      \resumeItem{Methodological Framework}
      {Proposed \textbf{Adaptive Masking Loss with Representation Consistency (AMRC)}, a novel optimization framework consisting of two mechanisms: an \textbf{Adaptive Masking Loss} that utilizes stochastic approximation to dynamically identify and retain discriminative temporal segments, and an \textbf{Embedding Similarity Penalty} that enforces geometric consistency between input embeddings and output manifolds to mitigate semantic inconsistency and representation collapse.}
      \resumeItem{Performance Achievement}
      {Demonstrated that the proposed constraints effectively restructure the optimization landscape, suppressing the learning of noise and irrelevant fluctuations. Validated the method's architecture-agnostic capability to enhance generalization and robustness across diverse backbones on standard multivariate benchmarks.}
    \resumeItemListEnd
    
    \resumeLongTitleSubheading
      {A Hybrid Framework for Evaluating and Enhancing Syntactic and Semantic Diversity}{Under Review}
      {First Author, Under Review}{Jan 2025 - July 2025}
    \resumeItemListStart
      \resumeItem{Theoretical Innovation \& Methodology}
      {Proposed a dual-stage hybrid generation paradigm to address the homogenization trade-off in low-resource NLG. Strategically decoupled syntactic skeleton generation from semantic refinement by utilizing probabilistic generative priors to capture diverse latent structures, followed by contextualization to ensure linguistic fidelity. This approach effectively reconstructs grammatically sound text while preserving the structural variance often lost in direct LLM generation.}
      \resumeItem{Metric Design: SynDiv}
      {Pioneered SynDiv, a reference-free metric for quantifying syntactic diversity without relying on gold standards. Innovatively modeled dependency parse trees as graphs and utilized Graph Laplacian spectral decomposition to extract fine-grained structural features from eigenvectors, enabling mathematically rigorous assessment of syntactic heterogeneity in unsupervised settings.}
      \resumeItem{Performance \& Insight}
      {Demonstrated that the hybrid framework induces Gaussian-like distributions in both semantic and syntactic feature spaces, significantly mitigating inherent training data biases. The augmented data yielded improvement on fine-grained sentiment classification (GoEmotions) establishing a strong correlation between structural diversity and model generalization boundaries.}
    \resumeItemListEnd
    
    \resumeSubheading
      {Thinking on the Move (ToM): Generative ABM for Urban Simulation}{}
      {NYU Center for Urban Science and Progress (Mentor: \textit{Prof. Takahiro Yabe})}{June 2025 - Present}
    \resumeItemListStart
      \resumeItem{Generative Agent Architecture}
      {Developed a novel generative agent-based modeling framework that utilizes fine-tuned open-source LLMs as the cognitive core of urban agents. By aligning the LLMs with granular mobility traces from Safegraph and Cuebiq data, the framework successfully enables agents to simulate complex, non-linear human decision-making processes and trajectory planning, surpassing the predictive accuracy of traditional gravity and discrete choice models.}
      \resumeItem{Predictive Analytics \& Causal Inference Framework}
      {Proposed a counterfactual reasoning mechanism within the simulation to assess the impact of spatial interventions. Enabled the system to generate high-fidelity what-if scenarios, moving beyond correlation-based forecasting to provide mechanistic insights into how urban infrastructure changes causally propagate through collective mobility behaviors and retail dynamics}
    \resumeItemListEnd
    
    \resumeSubheading
      {Intelligent School District Advisory Service}{Guided Study}
      {NYU Center for Urban Science and Progress (Mentor: \textit{Prof. Zhaoxi Zhang})}{Dec 2024 - Apr 2025}
    \resumeItemListStart
      \resumeItem{Data Engineering}
      {Architected a robust spatial-data infrastructure to integrate heterogeneous urban substrates, harmonizing disparate educational metrics with geospatial contexts. Addressed data inconsistency challenges through automated validity monitoring and advanced spatial indexing, establishing a unified analytical foundation for high-dimensional urban modeling.}
      \resumeItem{Geospatial Analysis}
      {Developed a context-aware advisory framework that synergizes GIS with LLMs. Implemented EduRAG to translate complex spatial-performance metrics into personalized insights, effectively bridging the gap between quantitative urban analytics and user-centric decision-making.}
    \resumeItemListEnd
  \resumeSubHeadingListEnd

\vspace{-3pt}
\section{Experience}
  \resumeSubHeadingListStart
    % \resumeSubheading
    % {Global Data Dive Competition - Skyline \& Sustainability \paper{https://docs.google.com/presentation/d/1Vf_OaO8iUelxuGhwhaDS-ypwzlFKihkM}}{New York, NY}{Best Technical Contribution Award Winner}{Feb 2025}
		% \resumeItemListStart
    %   \resumeItem{Advanced LLM Integration}
    %   {Implemented LLaMA-3-8B model with custom RAG architecture for real-time building energy efficiency analysis; optimized vector retrieval for 3D building data querying and developed prompt engineering techniques for domain-specific responses.}
    %   \resumeItem{Urban Sustainability Platform}
    %   {Created interactive 3D visualization of NYC buildings with energy compliance metrics; built end-to-end solution for policymakers and investors to assess sustainability impact through natural language queries and geospatial analytics.}
		% \resumeItemListEnd
		
    \resumeSubheading
    {SnowFox Technology Co., Ltd.}{Remote}
    {Founder \& Embedded System Architect}{Nov 2022 - Apr 2024}
    \resumeItemListStart
        \resumeItem{Resource-Constrained Sensor Fusion}
        {Architected a low-latency sensor fusion framework for wearable MEMS inertial arrays. Implemented a multi-rate Extended Kalman Filter with optimized fixed-point quaternion arithmetic, effectively resolving gimbal lock issues and drift accumulation on power-constrained microcontrollers.}
        \resumeItem{Edge-Cloud Biomechanical Analytics}
        {Developed a hierarchical motion analysis pipeline fusing Inverse Kinematics solvers with edge-deployed Bi-directional RNNs. Enabled real-time extraction of biomechanical features and movement classification, establishing a robust data synchronization protocol with AWS IoT Core for large-scale trajectory analysis.}
        \resumeItem{System Validation \& Deployment}
        {Validated the system's robustness and fault tolerance through extensive field testing in dynamic alpine environments, demonstrating stable performance under extreme temperature variations and high-impact motion scenarios.}
    \resumeItemListEnd

    \resumeSubheading
    {Autonomous Mobile Manipulation Platform (Smart Car Competition)}{Dalian, China}
    {Student Research Lead}{Dec 2022 - July 2023}
    \resumeItemListStart
        \resumeItem{Cyber-Physical System Design}
        {Led the development of an autonomous mobile robot integrating perception, decision-making, and actuation modules. Designed a holonomic motion control architecture for the Mecanum-wheeled chassis, utilizing customized PCBs to facilitate high-frequency communication between perception units and motor drivers.}
        \resumeItem{State Estimation \& Control}
        {Implemented a closed-loop control system fusing gyroscopic data with odometry via Kalman Filtering. Addressed non-linearities in mechanical transmission through PID parameter optimization, significantly enhancing trajectory tracking precision and dynamic stability against mechanical vibrations.}
        \resumeItem{Efficient Perception-Actuation Pipeline}
        {Deployed MobileNetV3 on edge processors for real-time target recognition. Engineered an automated manipulation logic that synchronizes visual feedback with mechanical arm inverse kinematics, achieving end-to-end autonomy in dynamic sorting tasks.}
    \resumeItemListEnd
    % \resumeSubheading{Dalian Jiaotong University{On-site}
    % {Research Assistant (Intern)}{Dec 2023 - May 2024}
    % \resumeItemListStart
    % \resumeItem{Structure Prototyping}
    % {Designed car chassis in SolidWorks, 3D-printed PLA compartments using Ultimaker.}
    % \resumeItem{Control System}
    % {Developed motherboard with RT1064, DRV8701 driver, voltage regulation; integrated ICM-20602, IMU, encoders data on RT-Thread position estimation, stable motion control with Kalman filter and PID tuning.}
    % \resumeItem{Object Recognition}
    % {Trained a MobileNetv2-based classifier on custom dataset; implemented distributed inference on edge devices with TensorRT optimization.}
    % \resumeItemListEnd
\resumeSubHeadingListEnd
    

%-----------PROJECTS-----------------
\vspace{-3pt}
\section{Projects}
\resumeSubHeadingListStart
\resumeProjectTitle{Multi-modal Context-aware RAG System}{Dec 2024 - June 2025}
\resumeItemContent{Developed RAG system handling text, images, and structured data with hybrid search and hallucination detection.}{LangChain, Pinecone, FAISS, HuggingFace Transformers, CLIP}

\resumeProjectTitle[\github{https://github.com/MazelTovy/image_segmentation_beamer}]{Small Object Image Segmentation}{Mar 2024 - June 2024}
\resumeItemContent{Active contour framework integrating improved YOLOv8, enhancing boundary precision and segmentation accuracy.}{PyTorch, OpenCV, YOLOv8, FPN, EMA, TensorRT, ONNX}

\resumeProjectTitle{Ecological Model for Fungal Biocontrol}{Aug 2022 - Mar 2023}
\resumeItemContent{Fungal growth management system for optimizing ecological balance and maximizing agricultural profitability.}{Tensorflow, RLib, MADDPG, PSO, PostgreSQL, Optuna, ArcGIS, Folium}

% \resumeProjectTitle{Quantitative Trading System}{Jan 2024 - Apr 2024}
% \resumeItemContent{Lightweight real-time price prediction and portfolio analysis system trained on historical stock markets data.}{Python, PyTorch, Ta-lib, TFT, XGBoost, EWMA, GluonTS, ONNX}

% \resumeProjectTitle[\github{https://github.com/MazelTovy/TinyTikTok}]{TinyTikTok}{July 2023 - Oct 2023}
% \resumeItemContent{Minimalistic TikTok backend with user authentication, uploading, liking, commenting, following, and messaging.}{Golang, Gin, MySQL, Redis, RabbitMQ, OSS, Nginx, Keycloak}
\resumeSubHeadingListEnd

\vspace{-3pt}
\section{Skills Summary}
	\resumeSubHeadingListStart
	\resumeSubItem{Languages}{~~~~~~Python, Rust, C++, R, SQL, Go, JAVA, \LaTeX, Swift}
	\resumeSubItem{Frameworks}{~~~~Scikit, PyTorch, LangChain, CUDA, Django, Spring Boot, Unity, NodeJS}
	\resumeSubItem{Tools}{~~~~~~~~~~~~~~SolidWorks, Docker, Kubernetes, ArcGIS, vLLM, Git, ONNX}
	\resumeSubItem{Platforms}{~~~~~~~Ubuntu, Kali, Raspberry Pi, ROS, NVIDIA Jetson, GCP, AWS}
	%\resumeSubItem{Soft Skills}{~~~~~~~Leadership, Project Management, Analytical Writing, Public Speaking, Communication, Time Management}
\resumeSubHeadingListEnd
% \section{Skills Summary}
% 	\resumeSubHeadingListStart
% 	\resumeSubItem{Languages}{~~~~~~Python, C++, Rust, SQL, CUDA, R, JavaScript, Go, JAVA}
% 	\resumeSubItem{ML/AI}{~~~~~~~~~~~~PyTorch, TensorFlow, JAX, HuggingFace, LangChain, LlamaIndex, ONNX, Scikit-learn}
% 	\resumeSubItem{HPC}{~~~~~~~~~~~~~~~CUDA, MPI, OpenMP, Ray, DeepSpeed, vLLM, TensorRT, SLURM, Distributed Training}
% 	\resumeSubItem{Data/Cloud}{~~~~~~Docker, Kubernetes, AWS, Azure ML, GCP, Hadoop, Spark, Airflow, Elasticsearch}
% \resumeSubHeadingListEnd

\vspace{-3pt}
\section{Honors and Awards}
\begin{description}[font=$\bullet$]
  \item{Spotlight Poster Presentation at NeurIPS 2025 - Sept, 2025 \paper{https://github.com/MazelTovy/AMRC}}
  \vspace{-5pt}
  \item{Best Technical Contribution Award - Global Data Dive Competition - Feb, 2025 \paper{https://docs.google.com/presentation/d/1Vf_OaO8iUelxuGhwhaDS-ypwzlFKihkM}}
  \vspace{-5pt}
  \item{Finalist of NYU CUSP Public Data Challenge - Oct, 2024 \paper{https://docs.google.com/presentation/d/1GGL3gzV5mJ0KLAir6hazS33j1yG3p9-z6fRmqLYKfB0}}
  \vspace{-5pt}
  \item{NYU CUSP Experiential Scholars - Sept, 2024}
  \vspace{-5pt}
  \item{First Prize of National Intelligent Car Competition - Aug, 2023 \github{https://github.com/MazelTovy/smart-car-18th}}
  \vspace{-5pt}
  \item{Bronze Medal in the China Collegiate Programming Contest (CCPC) - Oct, 2023}
\end{description}

\vspace{-3pt}
\section{References}
\resumeSubHeadingListStart
\resumeSubItem{Takahiro Yabe}{Assistant Professor at the Department of Technology Management and Innovation and the Center for Urban Science + Progress, New York University.}
\resumeSubItem{Joseph Chow}{Institute Associate Professor at the Department of Civil and Urban Engineering, Deputy Director of C2SMARTER University Transportation Center, New York University.}
\resumeSubItem{Zhaoxi Zhang}{Assistant Professor at the College of Design, Construction and Planning, University of Florida (formerly Postdoctoral Researcher at NYU CUSP during the mentorship period).}
\resumeSubHeadingListEnd

\end{document}